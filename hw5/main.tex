\documentclass{scrartcl}
\usepackage[margin=3cm]{geometry}
\usepackage{amsmath}
\usepackage{amssymb}
\usepackage{amsthm}
\usepackage{blindtext}
\usepackage{datetime}
\usepackage{fontspec}
\usepackage{graphicx}
\usepackage{kotex}
\usepackage{mathrsfs}
\usepackage{mathtools}
\usepackage{pgf,tikz,pgfplots}

\pgfplotsset{compat=1.15}
\usetikzlibrary{arrows}

\newcommand\Overline[2][0.8pt]{%
  \begin{tikzpicture}[baseline=(a.base)]
    \node[inner xsep=0pt,inner ysep=1.5pt] (a) {$#2$};
    \draw[line width= #1] (a.north west) -- (a.north east);
  \end{tikzpicture}
}
\newtheorem{theorem}{Theorem}

\setmainhangulfont{Noto Serif CJK KR}[
  UprightFont=* Light, BoldFont=* Bold,
  Script=Hangul, Language=Korean, AutoFakeSlant,
]
\setsanshangulfont{Noto Sans CJK KR}[
  UprightFont=* DemiLight, BoldFont=* Medium,
  Script=Hangul, Language=Korean
]
\setmathhangulfont{Noto Sans CJK KR}[
  SizeFeatures={
    {Size=-6,  Font=* Medium},
    {Size=6-9, Font=*},
    {Size=9-,  Font=* DemiLight},
  },
  Script=Hangul, Language=Korean
]
\title{MATH312: Homework 5 (due Nov. 8)}
\author{손량(20220323)}
\date{Last compiled on: \today, \currenttime}

\newcommand{\un}[1]{\ensuremath{\ \mathrm{#1}}}
\newcommand{\imag}{\operatorname{Im}}
\newcommand{\real}{\operatorname{Re}}
\newcommand{\Log}{\operatorname{Log}}
\newcommand{\Arg}{\operatorname{Arg}}
\DeclareMathOperator*{\Res}{Res}

\begin{document}
\maketitle

\section{Chapter 9 \#6}
Since polynomial functions are differentiable, \(D_1 f\) and \(D_2 f\) exists
for all \((x, y)\) that is not \((0, 0)\) as \(f\) is a quotient of two
polynomials, and \(x^2 + y^2 \not = 0\) for those points. We can also write
\begin{align*}
  (D_1 f) (0, 0)
  &= \lim_{h \to 0} \frac{f(h, 0) - f(0, 0)}{h}
  = \lim_{h \to 0} \frac{f(h, 0)}{h}
  = \lim_{h \to 0} \frac{0}{h}
  = 0 \\
  (D_2 f) (0, 0)
  &= \lim_{h \to 0} \frac{f(0, h) - f(0, 0)}{h}
  = \lim_{h \to 0} \frac{f(0, h)}{h}
  = \lim_{h \to 0} \frac{0}{h}
  = 0
\end{align*}
Thus, \((D_1 f) (x, y)\) and \((D_2 f) (x, y)\) exists for all \((x, y)\).

Now, suppose that \(f\) is continuous at \((0, 0)\). Then for all sequences
\(\{x_n\}\) that converges to \((0, 0)\), \(\{f(x_n)\}\) converges to \(f(0, 0)
= 0\) as \(n \to \infty\). Let \(a_n = (1 / n, 0)\) and \(b_n = (1 / n, 1 /
n)\). They converge to \((0, 0)\), and we can write
\begin{align*}
  \lim_{n \to \infty} f(a_n)
  &= \lim_{n \to \infty} f(1 / n, 0)
  = \lim_{n \to \infty} 0
  = 0 \\
  \lim_{n \to \infty} f(b_n)
  &= \lim_{n \to \infty} f(1 / n, 1 / n)
  = \lim_{n \to \infty} \frac{1}{2}
  = \frac{1}{2}
\end{align*}
Which is a contradiction. \(f\) is not continuous at \((0, 0)\).

\section{Chapter 9 \#7}
Fix \(\mathbf{p} \in E\). For all \(\mathbf{x} \in E\), we can write
\begin{align*}
  \mathbf{x} - \mathbf{p}
  = \sum^n_{i = 1} h_i \mathbf{e}_i
\end{align*}
where \(\mathbf{e}_i\; (i = 1, 2, \dots, n)\) are standard bases of
\(\mathbb{R}^n\). Let \(\mathbf{v}_0 = 0, \mathbf{v}_k = h_1 \mathbf{e}_1 +
\dots h_k \mathbf{e}_k\). Then we can write
\begin{align*}
  f(\mathbf{x}) - f(\mathbf{p})
  = \sum^n_{i = 1} (f(\mathbf{v}_i) - f(\mathbf{v}_{i - 1}))
\end{align*}
By triangle inequality,
\begin{align*}
  |f(\mathbf{x}) - f(\mathbf{p})|
  \le \sum^n_{i = 1} |f(\mathbf{v}_i) - f(\mathbf{v}_{i - 1})|
  \le \sum^n_{i = 1} |h_i| \left| \frac{f(\mathbf{v}_i)
    - f(\mathbf{v}_{i - 1})}{h_i} \right|
\end{align*}
By mean value theorem, there exists \(c_i\) between 0 and \(h_i\) such that
\begin{align*}
  (D_i f) (\mathbf{v}_{i - 1} + c_i \mathbf{e}_i)
  = \frac{f(\mathbf{v}_i) - f(\mathbf{v}_{i - 1})}{h_i}
\end{align*}
for all \(i = 1, \dots, n\). Since the partial derivatives are bounded, there
exists \(M_i \in \mathbb{R}\) such that \(|(D_i f) (\mathbf{x})| \le M_i\) for
all \(i = 1, \dots, n\) and \(\mathbf{x} \in E\). Then we can write
\begin{align*}
  |f(\mathbf{x}) - f(\mathbf{p})|
  \le \sum^n_{i = 1} |h_i| |(D_i f) (\mathbf{v}_{i - 1} + c_i \mathbf{e}_i)|
  \le \sum^n_{i = 1} |h_i| M_i
  \le |\mathbf{x} - \mathbf{p}| \sum^n_{i = 1} M_i
\end{align*}
For all \(\epsilon > 0\), we can take \(0 < \delta < \left( \sum^n_{i = 1} M_i
\right)^{-1}\). Then, \(|\mathbf{x} - \mathbf{p}| < \delta\) implies
\(|f(\mathbf{x}) - f(\mathbf{p})| < \epsilon\). Since the choice of
\(\mathbf{p}\) was arbitrary, \(f\) is continuous in \(E\).

\section{Chapter 9 \#8}
As \(f\) is differentiable, by theorem~9.17, we can write
\begin{align*}
  f'(\mathbf{x}) \mathbf{e}_i
  = (D_i f) (\mathbf{x})
\end{align*}
for all \(i = 1, \dots, n\) and \(\mathbf{x} \in E\). Here, we denote the
standard basis of \(\mathbb{R}^n\). Suppose that \(f\) has a local maximum at
\(\mathbf{p} = (p_1, \dots, p_n) \in E\). Using the definition of local
maximum, there exists an open ball \(B(\mathbf{p}, r) \subset E\) which is
centered at \(\mathbf{p}\) with radius \(r > 0\) such that for all \(\mathbf{x}
\in B(\mathbf{p}, r)\), \(f(\mathbf{x}) \le f(\mathbf{p})\). Let \(g_i(t) :=
f(\mathbf{p} + t\mathbf{e}_i)\) for \(i = 1, \dots, n\). Then \(g_i\) has a
local maximum at 0 for all \(i\), since for all \(t \in (-r, r)\), \(\mathbf{p}
+ t\mathbf{e}_i \in B(\mathbf{p}, r)\) so \(g_i(t) \le g_i(0)\). By
theorem~5.8, \(g_i'(0) = 0\) for all \(i\). Then, for all \(i\), we can write
\begin{align*}
  g_i'(0)
  = \lim_{t \to 0} \frac{g_i(t) - g_i(0)}{t}
  = \lim_{t \to 0} \frac{f(\mathbf{p} + t\mathbf{e}_i) - f(\mathbf{p})}{t}
  = (D_i f)(\mathbf{p}) = 0
\end{align*}
In conclusion, \(f'(\mathbf{p}) = 0\) as all the columns of
\([f'(\mathbf{p})]\) are zero.

\section{Chapter 9 \#9}
Let \(E_\mathbf{y} = \{\mathbf{x}\; |\; \mathbf{f}(\mathbf{x}) =
\mathbf{y}\}\). Consider a nonempty \(E_\mathbf{y}\). For all \(\mathbf{x} \in
E_\mathbf{y}\), there exists some open ball centered at \(\mathbf{x}\) of
radius \(r > 0\), \(B(\mathbf{x}, r)\) contained in \(E\). As open ball is
convex, by theorem~9.19 \(|\mathbf{f}(\mathbf{z}) - \mathbf{f}(\mathbf{x})| \le
0\) for all \(\mathbf{z} \in B(\mathbf{x}, r)\). By property of norm,
\(\mathbf{f}(\mathbf{x}) = \mathbf{f}(\mathbf{z})\) so \(\mathbf{z} \in
E_\mathbf{y}\). Thus, since the open ball is contained in \(E_\mathbf{y}\),
\(E_\mathbf{y}\) is open. For empty \(E_\mathbf{y}\), empty set is open so
\(E_\mathbf{y}\) is open for all \(\mathbf{y} \in \mathbb{R}^m\). Fix
\(\mathbf{p} \in E\). Then \(E_{\mathbf{f}(\mathbf{p})}\) is open, and \(E -
E_{\mathbf{f}(\mathbf(p))} = \bigcup_{\mathbf{y} \in \mathbb{R}^m -
\{\mathbf{f}(\mathbf{p})\}} E_\mathbf{y}\) is also open as it is a union of
open sets. Since \(E_{\mathbf{f}(\mathbf{p})}\) and \(E -
E_{\mathbf{f}(\mathbf{p})}\) are disjoint open sets in metric space \(E\), they
are separated. Sinec \(E\) is connected, it cannot be a union of two nonempty
separated sets, so one of \(E_{\mathbf{f}(\mathbf{p})}\) and \(E -
E_{\mathbf{f}(\mathbf{p})}\) is empty. Since \(E_{\mathbf{f(\mathbf{p})}}\)
cannot be empty by definition, \(E - E_{\mathbf{f}(\mathbf{p})}\) is empty,
which means that \(E = E_{\mathbf{f(\mathbf{p})}}\). In conclusion,
\(\mathbf{f}\) is constant.

\section{Chapter 9 \#13}
Let \(\mathbf{f}(t) = (f_1(t), f_2(t), f_3(t))\) and \(g(x_1, x_2, x_3) :=
x^2_1 + x^2_2 + x^2_3\). Then, we can write
\begin{align*}
  \lim_{\mathbf{h} \to 0} \frac{|g(\mathbf{x} + \mathbf{h})
    - g(\mathbf{x}) - (\nabla g(\mathbf{x})\mathbf{h})|}{|\mathbf{h}|}
  = \lim_{(h_1, h_2, h_3) \to (0, 0, 0)}
    \frac{|h^2_1 + h^2_2 + h^2_3|}{|\mathbf{h}|}
  = \lim_{\mathbf{h} \to 0} |\mathbf{h}|
  = 0
\end{align*}
Thus, \(g\) is differentiable. Then, as we can write \(|\mathbf{f}(t)| = 1\) as
\(g(\mathbf{f}(t)) = 1\), we can write
\begin{align*}
  g'(\mathbf{f}(t)) \mathbf{f}'(t)
  &= \begin{pmatrix}
    2f_1(t) & 2f_2(t) & 2f_3(t)
  \end{pmatrix} \begin{pmatrix}
    (D_1 f) (t) \\
    (D_2 f) (t) \\
    (D_3 f) (t)
  \end{pmatrix} \\
  &= 2 \begin{pmatrix}
    f_1(t) & f_2(t) & f_3(t)
  \end{pmatrix} \begin{pmatrix}
    (D_1 f) (t) \\
    (D_2 f) (t) \\
    (D_3 f) (t)
  \end{pmatrix}
  = 2 \mathbf{f}(t) \cdot \mathbf{f}'(t)
  = 0
\end{align*}
and get the desired result. Geometrically, \(\mathbf{f}\) draws a curve
embedded in a unit sphere, and the result implies that the curve's velocity
vector is normal to the radial vector. In other words, the velocity vector is
tanget to the spherical surface.

\section{Chapter 9 \#14}
\subsection{Solution for (a)}
For \((x, y) \not = (0, 0)\), we can write
\begin{align*}
  (D_1 f) (x, y)
  &= \frac{3x^2 (x^2 + y^2) - 2x (x^3)}{(x^2 + y^2)^2}
  = \frac{x^4 + 3x^2 y^2}{(x^2 + y^2)^2} \\
  (D_2 f) (x, y)
  &= \frac{-x^3 (2y)}{(x^2 + y^2)^2}
  = \frac{-2x^3 y}{(x^2 + y^2)^2}
\end{align*}
Also,
\begin{align*}
  (D_1 f) (0, 0)
  &= \lim_{h \to 0} \frac{f(h, 0) - f(0, 0)}{h}
  = \lim_{h \to 0} \frac{h - 0}{h}
  = 1 \\
  (D_2 f) (0, 0)
  &= \lim_{h \to 0} \frac{f(0, h) - f(0, 0)}{h}
  = \lim_{h \to 0} 0
  = 0
\end{align*}
For \((x, y) \not = (0, 0)\), we can write
\begin{align*}
  |(D_1 f) (x, y)|
  &= \left| \frac{(x^2 + y^2)^2 + y^2 (x^2 - y^2)}{(x^2 + y^2)^2} \right|
  \le 1 + \frac{|y^2 (x^2 - y^2)|}{(x^2 + y^2)^2}
  \le 1 + y^2 \cdot \frac{|x^2| + |y^2|}{(x^2 + y^2)^2} \\
  &= 1 + \frac{y^2}{x^2 + y^2}
  \le 2 \\
  |(D_2 f) (x, y)|
  &= \left| \frac{-2x^3 y}{(x^2 + y^2)^2} \right|
  = \left( \frac{x^2}{x^2 + y^2} \right) \left( \frac{|2xy|}{x^2 + y^2} \right)
  \le \frac{|2xy|}{x^2 + y^2}
  \le \frac{x^2 + y^2}{x^2 + y^2}
  = 1
\end{align*}
From this, we can conclude that the partial derivatives are bounded.

\subsection{Solution for (b)}
Let \(\mathbf{u} = (u_1, u_2)\). Then we can write
\begin{align*}
  (D_\mathbf{u} f) (0, 0)
  &= \lim_{t \to 0} \frac{f(t\mathbf{u}) - f(0, 0)}{t}
  = \lim_{t \to 0} \frac{f(tu_1, tu_2) - f(0, 0)}{t}
  = \lim_{t \to 0} \frac{1}{t} \frac{(tu_1)^3}{(tu_1)^2 + (tu_2)^2} \\
  &= \lim_{t \to 0} \frac{(tu_1)^3}{t^3}
  = u^3_1
\end{align*}
Since \(|\mathbf{u}| = 1\), \(|u_1| \le 1\) so the absolute value of
\((D_\mathbf{u} f) (0, 0)\) is at most 1.

\subsection{Solution for (c)}
Let \(\gamma(t) = (\gamma_1(t), \gamma_2(t))\). As \(\gamma\) is differentiable,
\(\gamma_1\) and \(\gamma_2\) are also differentiable. For \(t\) with
\(\gamma(t) \not = (0, 0)\), \(g(t)\) is differentiable as it is a quotient of
two differentiable functions, \((\gamma_1(t))^3\) and \((\gamma_1(t))^2 +
(\gamma_2(t))^2\). For \(t\) with \(\gamma(t) = (0, 0)\), we can write
\begin{align*}
  &\lim_{h \to 0} \frac{g(t + h) - g(t)}{h}
  = \lim_{h \to 0} \frac{f(\gamma(t + h)) - f(\gamma(t))}{h}
  = \lim_{h \to 0} \frac{f(\gamma(t + h))}{h} \\
  &= \lim_{h \to 0}
    \frac{(\gamma_1(t + h))^3}{h(\gamma_1(t + h))^2 + (\gamma_2(t + h))^2} \\
  &= \lim_{h \to 0}
    \left( \frac{\gamma_1(t + h) - \gamma_1(t)}{h} \right)^3
    \left[
      \left( \frac{\gamma_1(t + h) - \gamma_1(t)}{h} \right)^2
      + \left( \frac{\gamma_2(t + h) - \gamma_2(t)}{h} \right)^2
    \right]^{-1} \\
  &= \frac{(\gamma_1'(t))^3}{(\gamma_1'(t))^2 + (\gamma_2'(t))^2}
\end{align*}
Thus, \(g(t)\) is differentiable for all \(t\). Furthermore, we can write
\begin{align*}
  g'(t) = \begin{cases}
    \dfrac{(\gamma_1'(t))^3}{(\gamma_1'(t))^2 + (\gamma_2'(t))^2}
      & (\gamma(t) = (0, 0)) \\
    \dfrac{(\gamma_1(t))^4 \gamma_1'(t)
      + 3(\gamma_1(t))^2 (\gamma_2(t))^2 \gamma_1'(t)
      - 2(\gamma_1(t))^3 (\gamma_2(t)) \gamma_2'(t)}
      {[(\gamma_1(t))^2 + (\gamma_2(t))^2]^2}
      & (\gamma(t) \not = (0, 0))
  \end{cases}
\end{align*}
Now suppose that \(\gamma\) is continuously differentiable. For \(u\) such that
\(\gamma(u) = (0, 0)\), we can write
\begin{align*}
  \lim_{t \to u} g'(t)
  &= \lim_{t \to u} \frac{(\gamma_1(t))^4 \gamma_1'(t)
    + 3(\gamma_1(t))^2 (\gamma_2(t))^2 \gamma_1'(t)
    - 2(\gamma_1(t))^3 (\gamma_2(t)) \gamma_2'(t)}
    {[(\gamma_1(t))^2 + (\gamma_2(t))^2]^2} \\
  &= \frac{(\gamma_1'(u))^3}{(\gamma_1'(u))^2 + (\gamma_2'(u))^2}
\end{align*}
Thus, \(g'\) is continuous at \(u\). Since \(g'\) is obviously continuous at
points where \(\gamma\) is not \((0, 0)\), \(g'\) is continuous function so
\(g\) is continuously differentiable.

\subsection{Solution for (d)}
Using the hint, suppose that \(f\) is differentiable. Then, using \(\mathbf{u}
= (1 / \sqrt{2} , 1 / \sqrt{2})\), we can write
\begin{align*}
  (D_\mathbf{u} f) (0, 0)
  = \frac{1}{\sqrt{2}} (D_1 f) (0, 0) + \frac{1}{\sqrt{2}} (D_2 f) (0, 0)
  = \frac{1}{\sqrt{2}}
\end{align*}
However, according to the result of (b), the value should be \(1 /
(2\sqrt{2})\), which is a contradiction.

\section{Chapter 9 \#15}
\subsection{Solution for (a)}
As \((x^4 - y^2)^2 \ge 0\) holds, \((x^4 + y^2)^2 = (x^4 - y^2)^2 + 4x^4 y^2
\ge 4x^4 y^2\) also holds, proving the inequality.

For \((x, y) \not = (0, 0)\), since \(f\) is a sum of polynomial and quotient
of nonzero polynomials, \(f\) is continuous. Now let's show that \(f\) is
continuous at \((0, 0)\). Since polynomial is continuous everywhere, we only
have to show that \(4x^6 y^2 / (x^4 + y^2)^2\) is continuous at \((0, 0)\). We
can write
\begin{align*}
  \left| \frac{4x^6 y^2}{(x^4 + y^2)^2} \right|
  \le \left| \frac{4x^6 y^2}{4x^4 y^2} \right|
  = x^2
\end{align*}
using the result we have proven earlier. Fix \(\epsilon > 0\). Then by taking
\(\delta = \sqrt{\epsilon}\), \(|(x, y)| \le \delta\) implies \(x^2 \le
\epsilon\). Since the choice of \(\epsilon\) was arbitrary, \(4x^6 y^2 / (x^4 +
y^2)^2\) is continuous at \((0, 0)\). Thus, \(f\) is a continuous function.

\subsection{Solution for (b)}
By definition, \(g_\theta(0) = f(0, 0) = 0\). We can write
\begin{align*}
  g_\theta'(0)
  &= \lim_{t \to 0} \frac{f(t \cos \theta, t \sin \theta) - f(0, 0)}{t} \\
  &= \lim_{t \to 0}
    \frac{1}{t}
      \left[ t^2 - 2t^3 \cos^2 \theta \sin \theta
        - \frac{4t^4 \cos^6 \theta \sin^2 \theta}
        {(t^2 \cos^4 \theta + \sin^2 \theta)^2} \right] \\
  &= \lim_{t \to 0}
    \left[ t - 2t^2 \cos^2 \theta \sin \theta
      - \frac{4t^3 \cos^6 \theta \sin^2 \theta}
        {(t^2 \cos^4 \theta + \sin^2 \theta)^2}\right]
\end{align*}
For \(\theta\) with \(\sin \theta \not = 0\), the denominator of the fraction
term cannot be zero, so the limit is zero. For \(\theta\) with \(\sin \theta =
0\), the numerator is zero so the limit is zero again. Thus, \(g_\theta'(0) =
0\). With calculation we can see that
\begin{align*}
  g_\theta'(t)
  = 2t - 6t^2 \cos^2 \theta \sin \theta - \cos^6 \theta \sin^2 \theta
    \frac{16t^3 \sin^2 \theta}{(t^2 \cos^4 \theta + \sin^2 \theta)^3}
\end{align*}
and
\begin{align*}
  g_\theta''(0)
  &= \lim_{t \to 0} \frac{g_\theta'(t) - g_\theta'(0)}{t}
  = \lim_{t \to 0} \frac{g_\theta'(t)}{t} \\
  &= \lim_{t \to 0} \frac{1}{t}
    \left[ 2t - 6t^2 \cos^2 \theta \sin \theta - \cos^6 \theta \sin^2 \theta
      \frac{16t^3 \sin^2 \theta}
      {(t^2 \cos^4 \theta + \sin^2 \theta)^3} \right] \\
  &= \lim_{t \to 0}
    \left[ 2 - 6t \cos^2 \theta \sin \theta - \cos^6 \theta \sin^2 \theta
      \frac{16t^2 \sin^2 \theta}{(t^2 \cos^4 \theta + \sin^2 \theta)^3} \right]
\end{align*}
For \(\theta\) with \(\sin \theta \not = 0\), the denominator of the fraction
term cannot be zero, so the limit is zero. If \(\sin \theta = 0\), then the
numerator is zero so the limit is also zero. Thus \(g_\theta''(0) = 2\).

\subsection{Solution for (c)}
Plugging in the variables, we obtain \(f(x, x^2) = -x^4\). For all \(0 < r <
1\), we can take \(x = r / \sqrt{2}\) then \(|(x, x^2)| = \sqrt{x^2 + x^4} <
\sqrt{2x^2} = \sqrt{2} x < r\). As \(r > 0\), \(x > 0\) so \((x, x^2) \not =
(0, 0)\), and \(f(x, x^2) < f(0, 0) = 0\). In other words, for all \(r > 0\),
there exists a point \(p\) in \(B((0, 0), r)\), which is an open ball centered
at \((0, 0)\) whose radius is \(r\). This implies that \(f\) does not have a
local minimum at \((0, 0)\).

\end{document}
% vim: textwidth=79
