\documentclass{scrartcl}
\usepackage[margin=3cm]{geometry}
\usepackage{amsmath}
\usepackage{amssymb}
\usepackage{amsthm}
\usepackage{blindtext}
\usepackage{datetime}
\usepackage{fontspec}
\usepackage{graphicx}
\usepackage{kotex}
\usepackage{mathrsfs}
\usepackage{mathtools}
\usepackage{pgf,tikz,pgfplots}

\pgfplotsset{compat=1.15}
\usetikzlibrary{arrows}

\newcommand\Overline[2][0.8pt]{%
  \begin{tikzpicture}[baseline=(a.base)]
    \node[inner xsep=0pt,inner ysep=1.5pt] (a) {$#2$};
    \draw[line width= #1] (a.north west) -- (a.north east);
  \end{tikzpicture}
}
\newtheorem{theorem}{Theorem}

\setmainhangulfont{Noto Serif CJK KR}[
  UprightFont=* Light, BoldFont=* Bold,
  Script=Hangul, Language=Korean, AutoFakeSlant,
]
\setsanshangulfont{Noto Sans CJK KR}[
  UprightFont=* DemiLight, BoldFont=* Medium,
  Script=Hangul, Language=Korean
]
\setmathhangulfont{Noto Sans CJK KR}[
  SizeFeatures={
    {Size=-6,  Font=* Medium},
    {Size=6-9, Font=*},
    {Size=9-,  Font=* DemiLight},
  },
  Script=Hangul, Language=Korean
]
\title{MATH312: Homework 3 (due Oct. 4)}
\author{손량(20220323)}
\date{Last compiled on: \today, \currenttime}

\newcommand{\un}[1]{\ensuremath{\ \mathrm{#1}}}
\newcommand{\imag}{\operatorname{Im}}
\newcommand{\real}{\operatorname{Re}}
\newcommand{\Log}{\operatorname{Log}}
\newcommand{\Arg}{\operatorname{Arg}}
\DeclareMathOperator*{\Res}{Res}

\begin{document}
\maketitle

\section{Chapter 7 \#20}
By the theorem~7.26 there exists a sequence of polynomial \(\{P_n\}\) such that
\(P_n\) converges to \(f\) uniformly on \([0, 1]\). As \(P_n\) is polynomial on
\([0, 1]\), which is compact, so \(P_n\) is bounded for all \(n\). As
\(\{P_n\}\) converges uniformly, it is uniformly bounded. Thus, \(\{f P_n\}\)
converges uniformly to \(f^2\) on \([0, 1]\). Then, by the theorem~7.16, we can
write
\begin{align*}
  \int^1_0 f^2(x) dx = \lim_{n \to \infty} \int^1_0 f(x) P_n(x) dx
\end{align*}
We can write \(P_n(x)\) as \(a_{nk} x^k + \dots + a_{n1} x + a_{n0}\) for all
\(n\) where \(a_{nk}, a_{n, k - 1}, \dots, a_{n1}, a_{n0}\) are the
coefficients. Then, we can write
\begin{align*}
  \int^1_0 f(x) P_n(x) dx
  &= \int^1_0 f(x) (a_{nk} x^k + a_{n, k - 1} x^{k - 1} + \dots + a_{n1} x +
    a_{n0}) dx \\
  &= a_{nk} \int^1_0 f(x) x^k dx + a_{n, k - 1} \int^1_0 f(x) x^{k - 1} dx +
    \dots + a_{n0} \int^1_0 f(x) dx \\
  &= 0
\end{align*}
for all \(n\). Thus, we know that \(\int^1_0 f^2(x) dx = 0\). Then, as \(f\) is
continuous, \(f^2\) is alsp continuous and \(f^2 \ge 0\), so \(f(x) = 0\) for
all \(x \in [0, 1]\).

\section{Chapter 7 \#21}
Consider \(f \in \mathscr{A}\) such that \(f(e^{i\theta}) = e^{i\theta}\).
Then, \(f(x_1) \not = f(x_2)\) for \(x_1 \in K\) and \(x_2 \in K\) such that
\(x_1 \not = x_2\), so \(\mathscr{A}\) separates points on \(K\). Now consider
\(g \in \mathscr{A}\) such that \(g(e^{i\theta}) = 1\). Then, \(g(x) \not = 0\)
for all \(x \in K\), so \(\mathscr{A}\) vanishes at no point of \(K\).

Now, suppose that there exists \(h \in \bar{\mathscr{A}}\) such that
\(h(e^{i\theta}) = e^{-i\theta}\) where \(\bar{\mathscr{A}}\) is a uniform
closure of \(\mathscr{A}\). Fix \(\epsilon > 0\). Let \(p = e^{i\phi} \in K\).
Then we can write
\begin{align*}
  |h(e^{i\theta}) - h(p)|
  &= |e^{-i\theta} - e^{-i\phi}|
  = |(\cos(-\theta) - \cos(-\phi)) + i(\sin(-\theta) - \sin(-\phi))| \\
  &= |(\cos \theta - \cos \phi) - i(\sin \theta - \sin \phi)|
  = \sqrt{(\cos \theta - \cos \phi)^2 + (-\sin \theta + \sin \phi)^2} \\
  &= |(\cos \theta - \cos \phi) + i(\sin \theta - \sin \phi)|
  = |e^{i\theta} - p|
\end{align*}
Thus, if we take \(\delta > 0\) with \(\delta < \epsilon\), \(|e^{i\theta} - p|
< \delta\) implies \(|h(e^{i\theta}) - h(p)| < \epsilon\) for all \(e^{i\theta}
\in K\). Since the choice of \(\epsilon\) and \(p\) was arbitrary, \(h\) is
continuous on \(K\).

On the other hand, for all \(f \in \mathscr{A}\), we can write
\begin{align*}
  \int^{2\pi}_0 f(e^{i\theta}) e^{i\theta} d\theta
  &= \int^{2\pi}_0 e^{i\theta} \sum^N_{n = 0} c_n e^{in\theta}\; d\theta
  = \sum^N_{n = 0} c_n \int^{2\pi}_0 e^{i(n + 1)\theta} d\theta \\
  &= \sum^N_{n = 0} c_n \int^{2\pi}_0 (\cos ((n + 1)\theta) + i\sin ((n +
    1)\theta))\; d\theta \\
  &= \sum^N_{n = 0} c_n \left[ \frac{1}{n + 1} \sin ((n + 1)\theta)
    + \frac{-i}{n + 1} \cos ((n + 1)\theta) \right]^{2\pi}_0
  = 0
\end{align*}

As \(h \in \bar{\mathscr{A}}\), there exists \(\{h_n\} \subset \mathscr{A}\)
such that \(\{h_n\}\) converges to \(h\) uniformly. Then, as \(h_n \in
\mathscr{A}\) for all \(n\),
\begin{align*}
  \int^{2\pi}_0 h_n(e^{i\theta}) e^{i\theta} d\theta = 0
\end{align*}
However, for \(h\), we can write
\begin{align*}
  \int^{2\pi}_0 h(e^{i\theta}) e^{i\theta} d\theta
  = \int^{2\pi}_0 1\; d\theta
  = 2\pi
\end{align*}
Thus, we can write
\begin{align*}
  \lim_{n \to \infty} \int^{2\pi}_0 h_n(e^{i\theta}) e^{i\theta} d\theta
  = 0
  \not = 2\pi
\end{align*}
so from the contrapositive of theorem~7.16, the sequence of function of
\(\theta\) on \([0, 2\pi]\), \(\{h_n(e^{i\theta})\}\) does not converge
uniformly to \(h(e^{i\theta})\). Then, there exists \(\epsilon_0 \ge 0\),
sequence \(\{n_k\}\) and \(\{x_k\} \subset [0, 2\pi]\) such that
\(|h_{n_k}(e^{ix_k}) - h(e^{ix_k})| \ge \epsilon_0\) holds for sufficiently
large \(k\). Then, we can write \(y_k = e^{ix_k}\) and \(|h_{n_k}(y_k) -
h(y_k)| \ge \epsilon_0\). In conclusion, \(\{h_n\}\) does not converge
uniformly to \(h\), which is a contradiction and \(h \not \in
\bar{\mathscr{A}}\).

\section{Chapter 7 \#22}
Let \(P = \{x_0, \dots, x_n\}\) be a partition of \([a, b]\). Let \(g\) be a
function on \([a, b]\) which is defined as follows:
\begin{align*}
  g(t)
  = \frac{x_i - t}{\Delta x_i} f(x_{i - 1})
    + \frac{t - x_{i - 1}}{\Delta x_i} f(x_i)
\end{align*}
where \(t \in [x_{i - 1}, x_i]\). Fix \(\epsilon \in (0, 1)\). As \(f \in
\mathscr{R}(\alpha)\), there exists \(M > 0\) such that \(|f(x)| \le M\) for
all \(x \in [a, b]\). Also, there exists a partition \(P\) such that
\begin{align*}
  \sum^n_{i = 1} (M_i - m_i) \Delta\alpha_i
  < \frac{\epsilon^2}{2M}
\end{align*}
Then we can write
\begin{align*}
  \int^b_a |f - g|^2 d\alpha
  \le \sum^n_{i = 1} (M_i - m_i)^2 \Delta\alpha_i
  \le \sum^n_{i = 1} 2M(M_i - m_i) \Delta\alpha_i
  < \epsilon^2
  < \epsilon
\end{align*}
Thus, there exists continuous \(g\) such that
\begin{align*}
  \int^b_a |f - g|^2 d\alpha
  < \epsilon^2
  < \epsilon
\end{align*}
As \(g(t)\) is continuous on \([a, b]\), by theorem~7.26 there exists a
sequence of polynomials \(P_n\) such that \(\{P_n\}\) converges uniformly to
\(g\) on \([a, b]\). Thus, there exists \(N \in \mathbb{N}\) such that \(n \ge
N\) implies \(|g(x) - P_n(x)| < \epsilon\) for all \(x \in [a, b]\). Then
we can write
\begin{align*}
  \int^b_a |g - P_n|^2 d\alpha
  < \int^b_a \epsilon^2 d\alpha
  = \epsilon^2 (\alpha(b) - \alpha(a))
  < \epsilon (\alpha(b) - \alpha(a))
\end{align*}
Also, by the Schwarz~inequality, we can write
\begin{align*}
  \int^b_a |f - P_n|^2 d\alpha
  &= \int^b_a (f - g + g - P_n) \Overline{(f - g + g - P_n)} d\alpha \\
  &\le \int^b_a \left[ (|f - g|^2 + |g - P_n|^2) + (g - P_n)\Overline{(f - g)}
    + (f - g)\Overline{(g - P_n)}\right] d\alpha \\
  &\le \int^b_a |f - g|^2 d\alpha + \int^b_a |g - P_n|^2 d\alpha
    + 2\sqrt{\int^b_a |g - P_n|^2 d\alpha}\sqrt{\int^b_a |f - g|^2 d\alpha} \\
  &\le \left( \sqrt{\int^b_a |f - g|^2 d\alpha}
    + \sqrt{\int^a_b |g - P_n|^2 d\alpha} \right)^2 \\
  &= (\sqrt{\epsilon} + \sqrt{\epsilon(\alpha(b) - \alpha(a))})^2
  = (1 + \sqrt{\alpha(b) - \alpha(a)})^2 \epsilon
\end{align*}
Since the choice of \(\epsilon\) is arbitrary, we get the desired result.

\end{document}
% vim: textwidth=79
