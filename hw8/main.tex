\documentclass{scrartcl}
\usepackage[margin=3cm]{geometry}
\usepackage{amsmath}
\usepackage{amssymb}
\usepackage{amsthm}
\usepackage{blindtext}
\usepackage{datetime}
\usepackage{fontspec}
\usepackage{graphicx}
\usepackage{kotex}
\usepackage{mathrsfs}
\usepackage{mathtools}
\usepackage{pgf,tikz,pgfplots}

\pgfplotsset{compat=1.15}
\usetikzlibrary{arrows}

\newcommand\Overline[2][0.8pt]{%
  \begin{tikzpicture}[baseline=(a.base)]
    \node[inner xsep=0pt,inner ysep=1.5pt] (a) {$#2$};
    \draw[line width= #1] (a.north west) -- (a.north east);
  \end{tikzpicture}
}
\newtheorem{theorem}{Theorem}

\setmainhangulfont{Noto Serif CJK KR}[
  UprightFont=* Light, BoldFont=* Bold,
  Script=Hangul, Language=Korean, AutoFakeSlant,
]
\setsanshangulfont{Noto Sans CJK KR}[
  UprightFont=* DemiLight, BoldFont=* Medium,
  Script=Hangul, Language=Korean
]
\setmathhangulfont{Noto Sans CJK KR}[
  SizeFeatures={
    {Size=-6,  Font=* Medium},
    {Size=6-9, Font=*},
    {Size=9-,  Font=* DemiLight},
  },
  Script=Hangul, Language=Korean
]
\title{MATH312: Homework 8 (due Nov. 29)}
\author{손량(20220323)}
\date{Last compiled on: \today, \currenttime}

\newcommand{\un}[1]{\ensuremath{\ \mathrm{#1}}}
\newcommand{\imag}{\operatorname{Im}}
\newcommand{\real}{\operatorname{Re}}
\newcommand{\Log}{\operatorname{Log}}
\newcommand{\Arg}{\operatorname{Arg}}
\DeclareMathOperator*{\Res}{Res}

\begin{document}
\maketitle

\section{Problem \#1}
Suppose that \(\mathcal{A}\) is a \(\sigma\)-algebra. Then, by definition,
\(\mathcal{A}\) is closed under countable union, and \(\bigcup^\infty_{i = 1}
A_i \in \mathcal{A}\). Now, suppose that \(\mathcal{A}\) is closed under
countable increasing union. That is, for all \(A_1, A_2, \dots \in
\mathcal{A}\) such that \(A_1 \subset A_2 \subset \dots\), \(\bigcup^\infty_{i
= 1} A_i \in \mathcal{A}\) holds. For all \(B_1, B_2, \dots \in \mathcal{A}\),
let \(C_k = \bigcup^k_{i = 1} B_i\). Then, \(C_1 \subset C_2 \subset \dots\)
holds. As \(\mathcal{A}\) is closed under countable increasing unions,
\(\bigcup^\infty_{i = 1} C_i \in \mathcal{A}\). Since \(\bigcup^\infty_{i = 1}
C_i = \bigcup^\infty_{i = 1} A_i\), \(\mathcal{A}\) is a \(\sigma\)-algebra.

\section{Problem \#2}
As \(A \setminus B\) is disjoint with \(B\) and \(A \cap B\), we can write
\begin{align*}
  \mu(A \cup B)
  = \mu(A \setminus B) + \mu(B), \quad
  \mu(A)
  = \mu(A \setminus B) + \mu(A \cap B)
\end{align*}
From this, we obtain the desired result.
\begin{align*}
  \mu(A \cup B) + \mu(A \cap B)
  = \mu(A) + \mu(B)
\end{align*}

\section{Problem \#3}
It is known that every open sets in \(\mathbb{R}\) can be written as a
countable union of open intervals. Consider an open set \(U \subset
\mathbb{R}\). \(U\) can be written as \(U = \bigcup^\infty_{i = 1} (a_i,
b_i)\) where \(a_i < b_i\) for all \(i\), and since \(\mathcal{S}\) is a
\(\sigma\)-algebra and \((a_i, b_i) \in \mathcal{S}\) for all \(i\), \(U \in
\mathcal{S}\) also holds. Since the choice of \(U\) was arbitrary,
\(\mathcal{S}\) contains all open subsets of \(\mathbb{R}\). For all
\(\sigma\)-algebra \(\mathcal{A}\) containing all open sets, \(\mathcal{A}\)
contains all open intervals as open intervals are open in \(\mathbb{R}\). Then,
as \(\mathcal{S}\) is the smallest \(\sigma\)-algebra containing all open
intervals, \(\mathcal{S} \subset \mathcal{A}\). Thus, \(\mathcal{S}\) is the
smallest \(\sigma\)-algebra containing all open subsets of \(\mathbb{R}\). In
other words, \(\mathcal{S}\) is \(\mathcal{B}(\mathbb{R})\).

\section{Problem \#4}
Suppose that \(m^*((a, b) \cup (c, d)) = m^*((a, b)) + m^*((c, d))\) and \((a,
b) \cap (c, d) \not = \varnothing\). As \((a, b) \cup (c, d) = (\min \{a, c\},
\max \{b, d\})\) if \((a, b)\) and \(c, d\) are not disjoint, we can write
\begin{align*}
  \max \{b, d\} - \min \{a, c\}
  = (b - a) + (d - c)
\end{align*}
If \((a, b) \subset (c, d)\), \((a, b) \cup (c, d) = (c, d)\) so \(m^*((a, b)
\cup (c, d)) = d - c\), and the equality above does not hold, and by similar
argument, for \((c, d) \subset (a, b)\) the equality also does not hold.
Otherwise, if \(a < c < b < d\) then \((a, b) \cup (c, d) = (a, d)\) so
\(m^((a, b) \cup (c, d)) = d - a < (d - a) + (b - c)\), and the equality does
not hold. By similar argument the equality does not hold for \(c < a < d < b\)
case. Thus, \((a, b) \cap (c, d) \not = \varnothing\) is a contradiction, and
\((a, b) \cap (c, d) = \varnothing\) should hold.

Now, suppose that \((a, b) \cap (c, d) = \varnothing\). By a property of outer
measure, \(m^*((a, b) \cup (c, d)) \le m^*((a, b)) + m^*((c, d))\) holds.
Consider a collection of open intervals, \(\{I_n\}\) such that
\(\bigcup^\infty_{j = 1} I_j \supset (a, b) \cup (c, d)\). Let \(J_n := I_n
\cap (-\infty, a), K_n := I_n \cap (a, b), L_n := I_n \cap (b, \infty)\). Then
we can write
\begin{align*}
  \sum^\infty_{i = 1} l(I_i)
  \ge \sum^\infty_{i = 1} (l(J_i) + l(L_i)) + \sum^\infty_{i = 1} l(K_i)
  \ge m^*((a, b)) + m^*((c, d))
\end{align*}
Taking infinimum of both sides, we obtain \(m^*((a, b) \cup (c, d)) \ge m^*((a,
b)) + m^*((c, d))\). In conclusion, \(m^*((a, b) \cup (c, d)) = m^*((a, b)) +
m^*((c, d))\).

\section{Problem \#5}
If \(c = 0\), then \(cA \subset \{0\}\). Then, as \(cA \subset \{0\} \subset
(-r, r)\) for all \(r > 0\), \(\inf \{l((-r, r))\; |\; r \in (0, \infty)\} =
0\) holds, so \(m^*(cA) = 0 = |c| m^*(A)\). Now, suppose that \(c \not = 0\).
Consider a collection of open intervals, \(\{I_n\}\) such that
\(\bigcup^\infty_{j = 1} I_j \supset A\). Then, we can write
\begin{align*}
  c \left( \bigcup^\infty_{j = 1} I_j \right)
  = \bigcup^\infty_{j = 1} cI_j
  \supset cA
\end{align*}
and
\begin{align*}
  m^*(cA)
  \le \sum^\infty_{j = 1} l(cI_j)
  = |c| \sum^\infty_{j = 1} l(I_j)
\end{align*}
Taking infinimum of both sides, we obtain \(m^*(cA) \le |c| m^*(A)\). Now, let
\(B := cA, b := 1 / c\). Then, by the result we have proven earlier, \(m^*(bB)
\le |b| m^*(B)\), so \(m^*(cA) = m^*(B) \ge |b|^{-1} m^*(bB) = |c| m^*(A)\)
holds. In conclusion, \(m^*(cA) = |c| m^*(A)\).

\section{Problem \#6}
Since \(A \supset A \cap (-n, n)\), \(m^*(A) \ge m^*(A \cap (-n, n))\) holds
for all \(n \ge 1\), so \(m^*(A) \ge \lim_{n \to \infty} m^*(A \cap (-n, n))\).
Let \(B_n := A \cap ((-n, -n + 1] \cup [n - 1, n))\). Then, we can write
\begin{align*}
  m^*(A \cap (-n, n))
  = m^* \left( \bigcup^n_{i = 1} B_i \right)
  \le \sum^n_{i = 1} m^*(B_i)
\end{align*}
Now, consider a collection of open intervals, \(\{I_n\}\) such that
\(\cup^\infty_{j = 1} I_j \supset A \cap (-n, n)\). Let \(J_{ij} = I_j \cap
((-i, -i + 1] \cup [i - 1, i))\), then we can write
\begin{align*}
  \sum^\infty_{j = 1} l(I_j)
  \ge \sum^n_{i = 1} \sum^\infty_{j = 1} l(J_{ij})
  \ge \sum^n_{i = 1} m^*(B_i)
\end{align*}
By taking infinimum of both sides, we obtain
\begin{align*}
  m^*(A \cap (-n, n))
  \ge \sum^n_{i = 1} m^*(B_i)
\end{align*}
Thus, \(m^*(A \cap (-n, n)) = \sum^n_{i = 1} m^*(B_i)\). By taking limit of
both sides,
\begin{align*}
  \lim_{n \to \infty} m^*(A \cap (-n, n))
  = \sum^\infty_{i = 1} m^*(B_i)
  \ge m^* \left( \bigcup^\infty_{i = 1} B_i \right)
  = m^*(A)
\end{align*}
In conclusion, \(m^*(A) = \lim_{n \to \infty} m^*(A \cap (-n, n))\) and we get
the desired result.

\end{document}
% vim: textwidth=79
