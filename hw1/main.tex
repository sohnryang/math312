\documentclass{scrartcl}
\usepackage[margin=3cm]{geometry}
\usepackage{amsmath}
\usepackage{amssymb}
\usepackage{amsthm}
\usepackage{blindtext}
\usepackage{datetime}
\usepackage{fontspec}
\usepackage{graphicx}
\usepackage{kotex}
\usepackage{mathrsfs}
\usepackage{mathtools}
\usepackage{pgf,tikz,pgfplots}

\pgfplotsset{compat=1.15}
\usetikzlibrary{arrows}

\newcommand\Overline[2][0.8pt]{%
  \begin{tikzpicture}[baseline=(a.base)]
    \node[inner xsep=0pt,inner ysep=1.5pt] (a) {$#2$};
    \draw[line width= #1] (a.north west) -- (a.north east);
  \end{tikzpicture}
}
\newtheorem{theorem}{Theorem}

\setmainhangulfont{Noto Serif CJK KR}[
  UprightFont=* Light, BoldFont=* Bold,
  Script=Hangul, Language=Korean, AutoFakeSlant,
]
\setsanshangulfont{Noto Sans CJK KR}[
  UprightFont=* DemiLight, BoldFont=* Medium,
  Script=Hangul, Language=Korean
]
\setmathhangulfont{Noto Sans CJK KR}[
  SizeFeatures={
    {Size=-6,  Font=* Medium},
    {Size=6-9, Font=*},
    {Size=9-,  Font=* DemiLight},
  },
  Script=Hangul, Language=Korean
]
\title{MATH312: Homework 1 (due Sep. 18)}
\author{손량(20220323)}
\date{Last compiled on: \today, \currenttime}

\newcommand{\un}[1]{\ensuremath{\ \mathrm{#1}}}
\newcommand{\imag}{\operatorname{Im}}
\newcommand{\real}{\operatorname{Re}}
\newcommand{\Log}{\operatorname{Log}}
\newcommand{\Arg}{\operatorname{Arg}}
\DeclareMathOperator*{\Res}{Res}

\begin{document}
\maketitle

\section{Chapter 7 \#2}
Let \(f(x) := \lim_{n \to \infty} f_n(x), g(x) := \lim_{n \to \infty} g_n\).
Fix \(\epsilon > 0\). As both \(f_n\) and \(g_n\) converge uniformly, there
exists \(N \in \mathbb{N}\) such that \(n \geq N\) implies the following, for
all \(x \in E\).
\begin{align*}
  |f_n(x) - f(x)| < \frac{\epsilon}{2},\quad
  |g_n(x) - g(x)| < \frac{\epsilon}{2}
\end{align*}

Using triangle inequality, we can write
\begin{align*}
  |(f_n(x) + g_n(x)) - (f(x) - g(x))|
  &= |(f_n(x) - f(x)) + (g_n(x) - g(x))| \\
  &\leq |f_n(x) - f(x)| + |g_n(x) - g(x)|
  < \epsilon
\end{align*}

Since the choice of \(\epsilon\) here is arbitrary, there exists \(N\) such
that \(n \geq N\) implies \(|(f_n(x) + g_n(x)) - (f(x) - g(x))| < \epsilon\).
Thus, \(\{f_n + g_n\}\) converges uniformly.

As \(\{f_n\}\) is bounded for all \(n\), there exists \(\{A_n\} \subset
\mathbb{R}\) such that \(|f_n| \leq A_n\) for all \(n\). As \(f_n(x)\)
converges uniformly to \(f(x)\), three exists \(N \in \mathbb{N}\) such that
\(n \geq N\) implies \(|f_n(x) - f(x)| < \epsilon\) for all \(x \in E\). Then
by triangle inequality, \(|f(x)| - |f_n(x)| \leq |f_n(x) - f(x)| < \epsilon\)
holds for all \(n \geq N\) and \(x \in E\), so we can write
\begin{align*}
  |f(x)| \leq |f_N(x)| + \epsilon \leq A_N + \epsilon
\end{align*}
and \(f\) is a bounded function. We can also write
\begin{align*}
  |f_n(x)| \leq |f(x)| + \epsilon
\end{align*}
for all \(n \geq N\). As \(f\) is a bounded function, the following holds:
\begin{align*}
  |f_n(x)| \leq \max \{A_1, A_2, \dots, A_N, A_N + 2\epsilon\}
\end{align*}
Thus, \(\{f_n\}\) can be bounded by the same constant.

Let \(A, B\) be real numbers such that \(|f_n(x)| \leq A, |g_n(x)| \leq B\)
for all \(n\) and \(x \in E\). This can be done using the result we proved
earlier. We can write
\begin{align*}
  f_n(x) g_n(x) - f(x) g(x)
  &= \frac{1}{2} (f_n(x) - f(x))(g_n(x) - g(x)) \\
  &+ \frac{1}{2} (f_n(x) + f(x)) (g_n(x) - g(x))
\end{align*}
By triangle inequality,
\begin{align*}
  &|f_n(x) g_n(x) - f(x) g(x)| \\
  &\leq \left|\frac{1}{2} (f_n(x) - f(x))(g_n(x) - g(x)) \right|
  + \left| \frac{1}{2} (f_n(x) + f(x)) (g_n(x) - g(x)) \right| \\
  &\leq \frac{1}{2} |g_n(x) - g(x)| (|f_n(x)| + |f(x)|)
  + \frac{1}{2} |f_n(x) - f(x)| (|g_n(x)| + |g(x)|) \\
  &\leq A|f_n(x) - f(x)| + B|g_n(x) - g(x)|
\end{align*}
As \(\{f_n\}\) and \(\{g_n\}\) converge uniformly, there exists \(N\) such that
\(n \geq N\) implies \(|f_n(x) - f(x)| < \epsilon / (2A)\) and \(|g_n(x) -
g(x)| < \epsilon / (2B)\). Thus, \(\{f_n g_n\}\) converges uniformly.

\section{Chapter 7 \#3}
Consider \(f_n(x) = g_n(x) = x + (1 / n)\) on \(\mathbb{R}\). Fix \(\epsilon >
0\). Let \(N = \lceil 1 / \epsilon \rceil\). For all \(n \geq N\), we can write
\begin{align*}
  |f_n(x) - x| = \left| \frac{1}{n} \right|
  = \left| \left\lceil \frac{1}{\epsilon} \right\rceil^{-1} \right|
  \leq \epsilon
\end{align*}
and \(\{f_n\}\) and \(\{g_n\}\) converges uniformly to \(x\).

As \(1 / n\) and \(1 / n^2\) converges to zero as \(n\) tends to infinity,
\(f_n(x) g_n(x) = x^2 + 2x / n + 1 / n^2\) converges pointwisely to \(x^2\).
Consider a sequence \(x_n = n\). We can write
\begin{align*}
  |f_n(x_n) g_n(x_n) - x_n^2| = \left| \frac{2x_n}{n} + \frac{1}{n^2} \right|
  = \left| 2 + \frac{1}{n^2} \right| \geq 2
\end{align*}
Thus, \(\{f_n g_n\}\) does not converge uniformly.

\section{Chapter 7 \#5}
For \(x \in \mathbb{R}\), let \(N\) as follows:
\begin{align*}
  N = \begin{cases}
    1 & (x \leq 0) \\
    \left\lceil \frac{1}{x} \right\rceil & (x > 0)
  \end{cases}
\end{align*}
Fix \(\epsilon\). For \(n \geq N\), if \(x \leq 0\) then \(|f_n(x)| = 0 <
\epsilon\), and if \(x > 0\),
\begin{align*}
  x = \left( \frac{1}{x} \right)^{-1}
  \geq \left\lceil \frac{1}{x} \right\rceil^{-1} = \frac{1}{N} \geq \frac{1}{n}
\end{align*}
so \(|f_n(x)| = 0 < \epsilon\) and \(\{f_n\}\) converges to zero pointwisely,
which is a constant function, hence continuous.

However, for \(x_n = 1 / (n + 1/2)\), \(f_n(x_n) = 1 \geq 1\) for all \(n\).
Thus, \(\{f_n\}\) does not converge uniformly.

We can write
\begin{align*}
  \sum^\infty_{n = 1}
  = \sum^\infty_{n = 1} \sin^2 \left( \frac{\pi}{x} \right)
  \mathbf{I}_{\left[\frac{1}{n + 1}, \frac{1}{n}\right]}(x)
\end{align*}
where \(\mathbf{I}_A(x)\) is an inidcator function of set \(A\). As \((1 /
(n + 1), 1 / n)\), \((1 / (m + 1), 1 / m)\) are disjoint for \(n \not = m\)
and \(f_n(1 / k) = 0\) for all \(n \in \mathbb{N}\), \(k \in \mathbb{N}\), for
all \(x \in \mathbb{R}\), at most one of \(f_1(x), f_2(x), \dots\) is nonzero.
Thus, for all \(x\), only one of the terms of \(\sum f_n(x)\) is nonzero, so
\(\sum f_n(x)\) converges pointwisely for all \(x\), and since \(f_n(x) \geq
0\) for all \(x\), \(\sum f_n(x)\) converges absolutely.

\section{Chapter 7 \#7}
Using AM-GM inequality, for \(x \not = 0\) we can write
\begin{align*}
  |f_n(x)| = \left| \frac{x}{1 + nx^2} \right| \leq \frac{|x|}{2|x| \sqrt{n}}
  = \frac{1}{2\sqrt{n}}
\end{align*}
Since \(f_n(0) = 0\), \(f_n(x) \leq 1 / (2\sqrt{n})\) holds for all \(n\). For
all \(\epsilon > 0\), by taking \(N \in \mathbb{N}\) with \(N > 1 /
(4\epsilon^2)\), for all \(n \geq N\) the following holds for all \(x \in
\mathbb{R}\):
\begin{align*}
  |f_n(x)| \leq \frac{1}{2\sqrt{n}} \leq \frac{1}{2\sqrt{N}} < \epsilon
\end{align*}
Thus \(\{f_n\}\) converges uniformly to \(f(x) = 0\).

Taking the derivative of \(f_n(x)\),
\begin{align*}
  f'_n(x) = \frac{1 - nx^2}{(1 + nx^2)^2}
\end{align*}
For \(x \not = 0\),
\begin{align*}
  0 \leq f'_n(x) \leq \frac{\frac{1}{n} - x^2}{\frac{1}{n} + 2x^2 + nx^4}
  \leq \frac{\frac{1}{n}}{nx^4} < \frac{1}{n^2x^4}
\end{align*}
By sandwich theorem, \(\lim_{n \to \infty} f'_n(x) = 0\) as \(1 / (n^2x^4)\)
converges to zero as \(n\) tends to infinity. Since \(f'(x) = 0\), it is clear
that \(\lim_{n \to \infty} f'(x) = 0\) for \(x \not = 0\).
For \(x = 0\), \(f'_n(x) = 1\) so \(f'(x) \not = \lim_{n \to \infty} f'_n(x)\).

\section{Chapter 7 \#9}
By triangle inequality,
\begin{align*}
  |f_n(x_n) - f(x)| = |f_n(x_n) - f(x_n) + f(x_n) - f(x)|
  \leq |f_n(x_n) - f(x_n)| + |f(x_n) - f(x)|
\end{align*}
Fix \(\epsilon > 0\). Since \(\{f_n\}\) converges uniformly, there exists
\(N_1 \in N\) such that \(n \geq N_1\) implies \(|f_n(x) - f(x)| < \epsilon /
2\) for all \(x \in E\). Thus, \(|f_n(x_n) - f(x_n)| < \epsilon / 2\). By
theorem~7.12 in the book, \(f\) is continuous so by definition, for all
sequence \(\{x_n\} \subset E\) that converges to \(x\), \(f(x_n)\) converges to
\(f(x)\) as \(n\) tends to infinity. Thus, there exists \(N_2 \in \mathbb{N}\)
such that \(n \geq N_2\) implies \(|f(x_n) - f(x)| < \epsilon / 2\). Then, \(n
\geq \max \{N_1, N_2\}\) implies \(|f_n(x_n) - f(x)| < \epsilon\). Since the
choice of \(\epsilon\) here is arbitrary, we can conclude that \(\lim_{n \to
\infty} f_n(x_n) = f(x)\).

Consider a sequence of continuous functions, \(f_n: \mathbb{R} \to \mathbb{R}\)
defined as follows:
\begin{align*}
  f_n(x) = \begin{cases}
    \sin^2 \pi x & (x \in [n, n + 1]) \\
    0 & (x \not \in [n, n + 1])
  \end{cases}
\end{align*}
Fix \(\epsilon > 0\). Let \(\{x_n\} \subset \mathbb{R}\) be a sequence that
converges to \(x \in \mathbb{R}\). For \(x \leq 0\), there exists \(N \in
\mathbb{N}\) such that \(n \geq N\) implies \(|x_n - x| \leq |x| / 2\) as
\(x_n \to x\). Then, \(x_n \leq -|x| / 2\) so \(|f_n(x_n)| = 0 < \epsilon\).
For \(x > 0\), there exists \(N \in \mathbb{N}\) such that \(n \geq N\)
implies \(|x_n - x| < \left\lceil x \right\rceil - x + 1 / 2\) as \(x_n \to
x\). Then, \(n \geq \max\{N, \lceil x \rceil + 1 / 2\}\) implies \(|f_n(x_n)| =
0 < \epsilon\) as \(x_n < \lceil x \rceil + 1 / 2\) so \(n \geq N \geq \lceil
x \rceil + 1 / 2 > x_n\). Thus, \(\lim_{n \to \infty} f_n(x_n) = 0\) for all
\(\{x_n\} \subset \mathbb{R}\) such that \(x_n \to x \in \mathbb{R}\). However,
\(\{f_n\}\) does not converge to \(f(x) = 0\) uniformly. Consider \(x_n = n + 1
/ 2\). Then, \(f_n(x_n) = \sin^2 (\pi(n + 1 / 2)) = 1\) so \(|f_n(x_n)| \geq
1\). In conclusion, the converse is not true.

\section{Chapter 7 \#12}
Let \(h: (0, 1] \to \mathbb{R}\) and \(h_n: (0, 1] \to \mathbb{R}\) be defined
as follows:
\begin{align*}
  h(t) = \int^1_t f(x) dx, \quad h_n(t) = \int^1_t f_n(x) dx
\end{align*}
Fix \(\epsilon > 0\). We can write
\begin{align*}
  |h_n(t) - h(t)|
  &= \left| \int^1_t (f_n(x) - f(x)) dx \right|
  \leq \int^1_t |f_n(x) - f(x)| dx \\
  &\leq (1 - t) \sup_{t \leq x \leq 1} |f_n(x) - f(x)|
  \leq (1 - t) \sup_{t \in (0, \infty)} |f_n(x) - f(x)| \\
  &\leq \sup_{t \in (0, \infty)} |f_n(x) - f(x)|
\end{align*}
By theorem~7.9 there exists \(N \in \mathbb{N}\) such that \(\sup_{t \in (0,
\infty)} |f_n(x) - f(x)| < \epsilon\). Thus, \(\{h_n\}\) converges uniformly to
\(h\). Using the theorem~7.11,
\begin{align*}
  \lim_{t \to 0} \lim_{n \to \infty} h_n(t)
  = \lim_{n \to \infty} \lim_{t \to 0} h_n(t)
\end{align*}
and this can be written as
\begin{align}\label{sec6_zero_to_one}
  \int^1_0 f(x) dx = \lim_{n \to \infty} \int^1_0 f_n(x) dx
\end{align}

By sandwich theorem, for all \(x \in (0, \infty)\), \(\lim_{n \to \infty}
|f_n(x)| = |f(x)| \leq g(x)\). By the integral test of series, \(\int^\infty_1
f_n(x) dx\) converges if and only if \(\sum^\infty_{k = 1} f_n(k)\) converges.
Since we know that \(\int^\infty_0 g(x) dx\) converges, \(\int^\infty_1 g(x)
dx\) also converges as \(g(x) \geq 0\), and \(\sum^\infty_{n = 1} g(n)\) also
converges as a result of the integral test. By comparison test, we know that
\(\sum^\infty_{k = 1} |f_n(k)|\) converges for all \(n\), so \(\sum^\infty_{k =
1} f_n(k)\) converges absolutely. Thus, \(\int^\infty_1 f_n(x) dx\) converges
for all \(n\). The same logic can be applied for \(f(x)\), and \(\int^\infty_1
f(x) dx\) also converges. Let \(u \in (1, \infty)\). By triangle inequality,
\begin{align*}
  &\left| \int^\infty_1 f_n(x) dx - \int^\infty_1 f(x) dx \right| \\
  &= \left| \int^\infty_1 f_n(x) dx - \int^u_1 f_n(x) dx + \int^u_1 f_n(x) dx
  - \int^u_1 f(x) dx + \int^u_1 f(x) dx - \int^\infty_1 f(x) dx\right| \\
  &\leq \left| \int^\infty_1 f_n(x) dx - \int^u_1 f_n(x) dx \right|
  + \left| \int^u_1 f_n(x) dx - \int^u_1 f(x) dx \right|
  + \left| \int^u_1 f(x) dx - \int^\infty_1 f(x) dx \right|
\end{align*}
Fix \(\epsilon > 0\).
Since \(|f_n(x)| \leq g(x)\), we can write
\begin{align*}
  \left| \int^\infty_1 f_n(x) dx - \int^u_1 f_n(x) dx \right|
  = \left| \int^\infty_u f_n(x) dx \right|
  \leq \int^\infty_u |f_n(x)| dx
  \leq \int^\infty_u g(x) dx
\end{align*}
There exists some constant \(M > 1\) such that \(u \geq M\) implies
\(\int^\infty_u g(x) dx < \epsilon / 3\) by the definition of improper
integral. Also, we can write
\begin{align*}
  \left| \int^\infty_1 f(x) dx - \int^u_1 f(x) dx \right|
  = \left| \int^\infty_u f(x) dx \right|
  \leq \int^\infty_u |f(x)| dx
  \leq \int^\infty_u g(x) dx
\end{align*}
Then, \(u \geq M\) implies \(\left| \int^\infty_1 f_n(x) dx - \int^u_1 f_n(x)
dx \right| < \epsilon / 3\) and \(\left| \int^\infty_1 f_n(x) dx - \int^u_1
f_n(x) dx \right| < \epsilon / 3\).
Fix \(u\) to some real number greater or equal to \(M\). Then, by theorem~7.16,
\(\lim_{n \to \infty} \int^u_1 f_n(x) dx = \int^u_1 f(x) dx\), so there exists
\(N \in \mathbb{N}\) such that \(n \geq N\) implies \(\left| \int^u_1 f_n(x) dx
- \int^u_1 f(x) dx \right| < \epsilon / 3\). Thus, we can write
\begin{align*}
  \left| \int^\infty_1 f_n(x) dx - \int^\infty_1 f(x) dx \right| < \epsilon
\end{align*}
for \(n \geq N\), so
\begin{align}\label{sec6_one_to_infty}
  \int^\infty_1 f(x) dx = \lim_{n \to \infty} \int^\infty_1 f_n(x) dx
\end{align}
Using (\ref{sec6_zero_to_one}) and (\ref{sec6_one_to_infty}), we get the
desired result:
\begin{align*}
  \lim_{n \to \infty} \int^\infty_0 f_n(x) dx = \int^\infty_0 f(x) dx
\end{align*}
\end{document}
% vim: textwidth=79
